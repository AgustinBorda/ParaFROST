\documentclass[conference]{IEEEtran}
\IEEEoverridecommandlockouts
\usepackage{cite}
\usepackage{amsmath,amssymb,amsfonts}
\usepackage{algorithmic}
\usepackage{graphicx}
\usepackage{textcomp}
\usepackage{xcolor}
\usepackage{xspace}

%
\newcommand{\parafrost}{\textsc{ParaFROST}\xspace}
\newcommand{\minisat}{\textsc{MiniSat}\xspace}
\newcommand{\glucose}{\textsc{Glucose}\xspace}
\newcommand{\lingeling}{\textsc{Lingeling}\xspace}
\newcommand{\chaff}{\textsc{Chaff}\xspace}
\newcommand{\grasp}{\textsc{Grasp}\xspace}
\newcommand{\berkmin}{\textsc{BerkMin}\xspace}

% mute the stupid IEEEtrans bibtex error!
\makeatletter
\def\endthebibliography{%
	\def\@noitemerr{\@latex@warning{Empty `thebibliography' environment}}%
	\endlist
}
\makeatother

\begin{document}

\title{ParaFROST at the SAT Race 2020
\thanks{This work is part of the GEARS project with project number TOP2.16.044, which is (partly) financed by the Netherlands Organisation for Scientific Research (NWO).}
}

\author{\IEEEauthorblockN{Muhammad Osama and Anton Wijs}
	\IEEEauthorblockA{Department of Mathematics and Computer Science\\
		Eindhoven University of Technology, 5600 MB Eindhoven, The Netherlands\\
	}
}

\maketitle

\section{Introduction}
This paper presents a brief description to our solver \parafrost{} which stands for \emph{Parallel Formal Reasoning Of SaTisfiability}. Our solver is based on state-of-the-art CDCL search~\cite{grasp,minisat,glucose}, integrated with preprocessing as presented in SIGmA (SAT sImplification on GPU Architectures)~\cite{sigmaTacas,sigmaIfm} and a new technique called Parallel Decision Making (PDM)~\cite{pdcl}. The submitted version only allows a single-threaded CPU execution.

\parafrost provides easy-to-use infrastructure for SAT solving and/or preprocessing with optimized data structures for both CPU/GPU architectures, and fine-tuned heuristic parameters. The \emph{Parallel} keyword in \parafrost intuitively means that SAT simplifications can be fully executed on variables in parallel as described in~\cite{sigmaTacas} using the Least Constrained Variable Elections (LCVE) algorithm while periodically, via the PDM procedure~\cite{pdcl}, the solver is capable of making decisions that can be assigned and propagated in parallel as well. Choosing variables to preprocess or decisions relies heavily on \emph{freezing} (that is where FROST originated) mutually independent variables according to some logical properties.  
%========== references ===========
\bibliographystyle{IEEEtran}
\bibliography{IEEEabrv,pf_refs}

\end{document}
